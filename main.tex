\documentclass{beamer}
\usepackage{graphicx}
\usepackage{listings}
\usepackage{amsmath}
\usepackage{amssymb}
\usepackage{algorithm}
\usepackage{algpseudocode}
\usepackage{multimedia}
\usetheme{Berkeley}
\definecolor{mycolor}{RGB}{120, 130, 145} 
\setbeamercolor{structure}{fg=mycolor}
\setbeamercolor{background canvas}{bg=white!90} 
\usepackage{tikz}
\usetikzlibrary{trees, shapes, arrows}
\tikzstyle{block} = [rectangle, draw, fill=white!20, text width=6em, text centered, rounded corners, minimum height=4em]
\tikzstyle{line} = [draw, -latex']
\tikzstyle{cloud} = [draw, ellipse, fill=white!20, minimum height=2em]
\logo{\includegraphics[height=1cm]{iitr-logo.jpg}}

\title{Implementing Finite Element Approximations}
\subtitle{From the perspective of Algorithmic Implementation}
\institute{Department of Mathematics, IIT Roorkee}
\date{\today}
\author{Ved Umrajkar}

\definecolor{codegreen}{rgb}{0,0.6,0}
\definecolor{codegray}{rgb}{0.5,0.5,0.5}
\definecolor{codepurple}{rgb}{0.58,0,0.82}
\definecolor{backcolour}{rgb}{0.95,0.95,0.92}

\lstdefinestyle{mystyle}{
    backgroundcolor=\color{backcolour},   
    commentstyle=\color{codegreen},
    keywordstyle=\color{magenta},
    numberstyle=\tiny\color{codegray},
    stringstyle=\color{codepurple},
    basicstyle=\ttfamily\footnotesize,
    breakatwhitespace=false,         
    breaklines=true,                 
    captionpos=b,                    
    keepspaces=true,                 
    numbers=left,                    
    numbersep=5pt,                  
    showspaces=false,                
    showstringspaces=false,
    showtabs=false,                  
    tabsize=2
}

\lstset{style=mystyle}

\begin{document}

\begin{frame}
\titlepage
\end{frame}

\begin{frame}{Outline}
\tableofcontents
\end{frame}

\section{Problem Formulation and Discretization}

\begin{frame}{The Poisson Equation and Weak Formulation}
\begin{itemize}
    \item Problem: $-\nabla^2 u = f$ in $\Omega \subset \mathbb{R}^d$, subject to boundary conditions
    \item Weak form: Find $u \in H^1_0(\Omega)$ such that
    \begin{align}
    a(u,v) &= \ell(v) \quad \forall v \in H^1_0(\Omega)\\
    a(u,v) &= \int_\Omega \nabla u \cdot \nabla v \, d\Omega\\
    \ell(v) &= \int_\Omega f \cdot v \, d\Omega
    \end{align}
    \item Galerkin approximation: Find $u_h \in V_h \subset H^1_0(\Omega)$ s.t.
    \begin{align}
    a(u_h, v_h) = \ell(v_h) \quad \forall v_h \in V_h
    \end{align}
    where $V_h$ is a finite-dimensional subspace with basis $\{\phi_1, \phi_2, \ldots, \phi_n\}$
\end{itemize}
\end{frame}

\begin{frame}{FEM Linear System}
\begin{itemize}
    \item Approximation: $u_h = \sum_{j=1}^n u_j \phi_j(x)$
    \item Linear system: $\mathbf{A}\mathbf{u} = \mathbf{b}$
    \item Stiffness matrix: $A_{ij} = \int_\Omega \nabla \phi_i \cdot \nabla \phi_j \, d\Omega$
    \item Load vector: $b_i = \int_\Omega f \cdot \phi_i \, d\Omega$
\end{itemize}

\begin{algorithm}[H]
\caption{Finite Element Method}
\begin{algorithmic}[1]
\State Discretize domain $\Omega$ into elements
\State Choose basis functions $\{\phi_i\}_{i=1}^n$
\State Assemble stiffness matrix $A_{ij} = \int_\Omega \nabla \phi_i \cdot \nabla \phi_j \, d\Omega$
\State Assemble load vector $b_i = \int_\Omega f \cdot \phi_i \, d\Omega$
\State Apply boundary conditions to modify $A$ and $b$
\State Solve linear system $Au = b$
\State Reconstruct solution $u_h = \sum_{j=1}^n u_j \phi_j(x)$
\end{algorithmic}
\end{algorithm}
\end{frame}

\section{Numerical Integration Methods}

\begin{frame}{Numerical Quadrature Formulations}
\begin{table}
\begin{tabular}{|l|p{6cm}|}
\hline
\textbf{Method} & \textbf{Formula} \\
\hline
Gauss-Legendre & $\int_{-1}^1 f(x) \, dx \approx \sum_{i=1}^n w_i f(x_i)$ \\
\hline
Newton-Cotes & $\int_a^b f(x) \, dx \approx (b-a) \sum_{i=0}^n A_i f(x_i)$ \\
\hline
Trapezoidal & $\int_a^b f(x) \, dx \approx \frac{b-a}{2}[f(a) + f(b)]$ \\
\hline
Simpson's & $\int_a^b f(x) \, dx \approx \frac{b-a}{6}[f(a) + 4f(\frac{a+b}{2}) + f(b)]$ \\
\hline
\end{tabular}
\end{table}
\end{frame}
\begin{frame}{Gauss Quadrature Points and Weights}
\begin{table}
\centering
\begin{tabular}{|c|c|c|c|}
\hline
\textbf{Order} & \textbf{Points} $x_i$ & \textbf{Weights} $w_i$ & \textbf{Exact for} \\
\hline
1 & 0 & 2 & $P_1$ \\
\hline
2 & $\pm\frac{1}{\sqrt{3}}$ & 1 & $P_3$ \\
\hline
3 & 0, $\pm\sqrt{\frac{3}{5}}$ & $\frac{8}{9}$, $\frac{5}{9}$ & $P_5$ \\
\hline
\end{tabular}
\caption{Gauss-Legendre quadrature on $[-1,1]$}
\end{table}

\begin{algorithm}[H]
\caption{Numerical Integration for FEM}
\begin{algorithmic}[1]
\State Map element to reference element
\State Select quadrature rule based on polynomial degree of integrand
\State Apply quadrature formula to compute integral
\State Map result back to original element
\end{algorithmic}
\end{algorithm}
\end{frame}

\begin{frame}{Error Analysis for Numerical Integration}
\begin{itemize}
    \item \textbf{Gauss-Legendre} ($n$ points): 
    \begin{align}
    E_n(f) = \frac{2^{2n+1}(n!)^4}{(2n+1)[(2n)!]^3}f^{(2n)}(\xi), \quad \xi \in (-1,1)
    \end{align}
    Exact for polynomials of degree $\leq 2n-1$
    
    \item \textbf{Trapezoidal Rule}:
    \begin{align}
    E_T(f) = -\frac{(b-a)^3}{12}f''(\xi), \quad \xi \in (a,b)
    \end{align}
    
    \item \textbf{Simpson's Rule}:
    \begin{align}
    E_S(f) = -\frac{(b-a)^5}{2880}f^{(4)}(\xi), \quad \xi \in (a,b)
    \end{align}
    
    \item For FEM with $P_k$ Polynomial Space, we need quadrature exact for polynomials of degree $\geq 2k$
\end{itemize}
\end{frame}

\section{Basis Functions and Convergence}


\begin{frame}{Density of Polynomials \& Weierstrass Approximation Theorem}
\begin{theorem}[Weierstrass Approximation Theorem]
If $f$ is a continuous real-valued function on $[a,b]$, then for every $\varepsilon > 0$, there exists a polynomial $p$ such that
\[ \sup_{x \in [a,b]} |f(x) - p(x)| < \varepsilon \]
\textbf{Similarly, $\overline{P{\infty}(\Omega)} = C(\Omega)$ for bounded domains}
\end{theorem}
\begin{example}
Consider $f(x) = |x|$ on $[-1,1]$ and its polynomial approximations:
\begin{align}
p_n(x) = \sum_{k=0}^n \frac{(-1)^k(2k)!}{2^{2k}(k!)^2(2k+1)}x^{2k+1}
\end{align}
\end{example}
\end{frame}

\begin{frame}
  \frametitle{Basis Functions for Function Approximation}
  
  \begin{block}{Commonly Used Polynoomial Basis Functions}
    \begin{itemize}
      \item \textbf{Monomial Basis:} $\{1, x, x^2, x^3, \ldots, x^n\}$
        \begin{itemize}
          \item Simplest basis, but often ill-conditioned for higher degrees
        \end{itemize}
      
      \item \textbf{Legendre Polynomials:} $\{P_0(x), P_1(x), P_2(x), \ldots, P_n(x)\}$
        \begin{itemize}
          \item Recurrence: $(k+1)P_{k+1}(x) = (2k+1)xP_k(x) - kP_{k-1}(x)$
          \item $P_0(x) = 1$, $P_1(x) = x$
          \item Orthogonal on $[-1,1]$ with weight $w(x) = 1$
        \end{itemize}
      
      \item \textbf{Chebyshev Polynomials:} $\{T_0(x), T_1(x), T_2(x), \ldots, T_n(x)\}$
        \begin{itemize}
          \item Recurrence: $T_{k+1}(x) = 2xT_k(x) - T_{k-1}(x)$
          \item $T_0(x) = 1$, $T_1(x) = x$
          \item Orthogonal on $[-1,1]$ with weight $w(x) = \frac{1}{\sqrt{1-x^2}}$
          \item Minimize maximum error (minimax property)
        \end{itemize}
      
    \end{itemize}
  \end{block}
\end{frame}

\begin{frame}
  \frametitle{Basic Algorithm to Compute the $L^2$-Projection}
  
  \begin{block}{Algorithm for $L^2$-Projection}
    The following algorithm computes the $L^2$-projection $P_h f$:
    
    \begin{enumerate}
      \item Create a mesh with $n$ elements on the interval $I$ and define $V_h$.
      
      \item Compute the $(n+1) \times (n+1)$ matrix $\mathbf{M}$ and the $(n+1) \times 1$ vector $\mathbf{b}$, with entries:
      \begin{align}
        M_{ij} = \int_I \varphi_j \varphi_i \, dx, \quad b_i = \int_I f \varphi_i \, dx
      \end{align}
      
      \item Solve the linear system:
        \mathbf{M} \boldsymbol{\xi} = \mathbf{b}
      
      \item Set:
      \begin{align}
        P_h f = \sum_{j=0}^n \xi_j \varphi_j
      \end{align}
    \end{enumerate}
  \end{block}
\end{frame}



\begin{frame}{Demo}
    \begin{figure}
        \centering
        \movie[width=0.5\linewidth,height=0.4\linewidth,autostart,loop]{}{l2-gif.gif}
        \caption{L2 Projections of $|x|$ with different basis functions}
        \label{fig:l2-projections}
    \end{figure}
\end{frame}

\begin{frame}{Motivation for the Finite Element Method}
For any $\varepsilon > 0$, $\exists N$ such that $\forall n \geq N$:
\begin{align}
\|f - p_n\|_{\infty} < \varepsilon
\end{align}
This theoretical foundation enables FEM through:
\begin{itemize}
\item Polynomial spaces $P[\Omega]$ can approximate any solution $u \in C(\Omega)$
\item Complex domains $\Omega$ can be partitioned into simple elements
\item Local polynomial basis functions provide computationally efficient approximations
\item The variational formulation transforms PDEs into algebraic systems
\item Systematic refinement guarantees convergence to the true solution
\end{itemize}
\end{frame}
\begin{frame}{Basis Functions and Their Properties}
\begin{itemize}
    \item \textbf{Linear elements} ($P_1$):
    \begin{align}
    \phi_i(x_j) = \delta_{ij}, \quad \phi_i \in P_1
    \end{align}
    In 1D: $\phi_i(x) = \begin{cases}
        \frac{x-x_{i-1}}{x_i-x_{i-1}} & \text{if } x \in [x_{i-1}, x_i] \\
        \frac{x_{i+1}-x}{x_{i+1}-x_i} & \text{if } x \in [x_i, x_{i+1}] \\
        0 & \text{otherwise}
        \end{cases}$
        
    \item \textbf{Quadratic elements} ($P_2$):
    \begin{align}
    \phi_i(x_j) = \delta_{ij}, \quad \phi_i \in P_2
    \end{align}
    
    \item \textbf{Convergence rates}:
    \begin{align}
    \|u - u_h\|_{L^2(\Omega)} &\leq Ch^{k+1}|u|_{H^{k+1}(\Omega)} \\
    \|u - u_h\|_{H^1(\Omega)} &\leq Ch^{k}|u|_{H^{k+1}(\Omega)}
    \end{align}
    where $k$ is the polynomial degree of basis functions
\end{itemize}
\end{frame}

\begin{frame}{Implementation of 1D Linear System Assembly}
\begin{algorithm}[H]
\caption{Efficient Assembly for 1D Linear Elements}
\begin{algorithmic}[1]
\State Create mesh with $n$ elements: $x = \{x_0, x_1, \ldots, x_n\}$
\State Set $h = 1/n$ (uniform mesh)
\State Initialize stiffness matrix diagonals: 
      $\text{main} = [2/h, 2/h, \ldots, 2/h]$ (length $n+1$)
\State Initialize off-diagonals: 
      $\text{off} = [-1/h, -1/h, \ldots, -1/h]$ (length $n$)
\State Create sparse matrix $A$ using diagonals
\State Compute load vector using quadrature:
      $b_i = \int_\Omega f \cdot \phi_i \, d\Omega$
\State Apply boundary conditions
\State Solve system $Au = b$
\end{algorithmic}
\end{algorithm}


\end{frame}
\begin{frame}{}
\begin{algorithm}[H]
\caption{Quadrature for Load Vector (Quadratic Elements)}
\begin{algorithmic}[1]
\For{each element $e$ from $1$ to $n$}
    \State $x_1, x_3 \gets$ element endpoints
    \State $gp_1, gp_2, gp_3 \gets x_1 + 0.1127h, x_1 + 0.5h, x_1 + 0.8873h$ \Comment{Gauss points}
    \State $f_{gp} \gets f(gp_1, gp_2, gp_3)$ \Comment{Evaluate source at Gauss points}
    \State $b_e \gets h/6 \cdot [f_{gp}[0] + 0.5f_{gp}[1] + 0.1f_{gp}[2], \ldots]$ \Comment{Element load}
    \State Assemble $b_e$ into global $b$
\EndFor
\end{algorithmic}
\end{algorithm}
    
\end{frame}

\section{Results and Demonstration}

\begin{frame}{Results and Demonstration}
    
\end{frame}
\subsection{Live Demo}
\begin{frame}{Live Demo}
    \begin{itemize}
        \item Code can be found at the following link
        \item Packages Used: Numpy, Scipy, scikit-fem, matplotlib, seaborn
    \end{itemize}
    
\end{frame}

\begin{frame}{References}
\begin{thebibliography}{5}

\bibitem{scikit-fem} G. Gustafsson, T. Ruubel, \emph{scikit-fem: A Python Package for Finite Element Assembly}, JOSS, 2020.

\bibitem{larson2013}
Larson, M.G., Bengzon, F.: The Finite Element Method: Theory, Implementation, and Applications. Springer Publishing Company, Incorporated (2013)

\end{thebibliography}
\end{frame}

\begin{frame}{}

    
\end{frame}

\end{document}